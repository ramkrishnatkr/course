\documentclass[pdftex]{beamer}
\usetheme{metropolis}

\usepackage[english]{babel}
\usepackage[latin1]{inputenc}
\usepackage{times}
\usepackage[T1]{fontenc}
\usepackage{fancyvrb}
\usepackage{listings}
\begin{document}
\lstset{language=C, escapeinside={(*@}{@*)}, numbers=left,
  basicstyle=\tiny, showspaces=false, showtabs=false, showstringspaces=false}

\title{A Look Inside FreeBSD with DTrace}
\subtitle{Introduction and Tutorial Overview}
\author[shortname]{George V. Neville-Neil \and Robert N. M. Watson}

\begin{frame}
  \titlepage
\end{frame}

\begin{frame}
  \frametitle{Objectives}
  \begin{itemize}
  \item Understand key kernel concepts
  \item Become comfortable with DTrace
    \begin{itemize}
    \item Terminology
    \item Basic Usage
    \item Advanced Scripting
    \end{itemize}
  \item Explore on your own
  \end{itemize}
\end{frame}

%
% This interactive whiteboarding exercise should take five minutes or less:
% ask about places where OSes are used, what services they provide, what other
% problems they can think of.  Where have OSes let them down?  What is in an OS?
%

\begin{frame}
  \begin{huge}
  \begin{center}
    What is an operating system?
  \end{center}
  \end{huge}

  \bigskip

  \begin{center}
  \begin{tiny}
    Whiteboarding exercise
  \end{tiny}
  \end{center}
\end{frame}

\begin{frame}
  \frametitle{What is an operating system?}

  \begin{large}
  \begin{center}
    [An OS is] low-level software that supports \\
    a computer's basic functions, such as \\
    scheduling tasks and controlling peripherals. \\
  \smallskip
  \small{- Google hive mind}
  \end{center}
  \end{large}
\end{frame}

\begin{frame}
  \frametitle{General-purpose operating systems}

  ...  are for general-purpose computers

  \begin{itemize}
    \item Servers, workstations, mobile devices
    \item Run `applications' -- i.e., software unknown at design time
    \item Abstract the hardware, provide `class libraries'
    \item E.g., Windows, Mac OS X, Android, iOS, Linux, FreeBSD, ...
  \end{itemize}

  \pause

  \medskip

  \begin{description}
    \item[\textbf{Userspace}]

      \smallskip

      Local and remote shells, management tools, daemons

      \smallskip

      Run-time linker, system libraries, tracing facilities

    \pause

    \item[] {\tiny - - - - \textit{system-call interface} - - - -}

    \pause

    \item[\textbf{Kernel}]

      System calls, hypercalls, remote procedure call (RPC)

      \smallskip

      Processes, filesystems, IPC, sockets, management

      \smallskip

      Drivers, packets/blocks, protocols, tracing, virtualisation

      \smallskip

      VM, malloc, linker, scheduler, threads, timers, tasks, locks

  \end{description}

  \pause

  \medskip

  Continuing disagreement on whether distributed-filesystem servers and
  window systems `belong' in userspace or the kernel
\end{frame}

\begin{frame}
  \frametitle{What does an operating system do?}

  \begin{itemize}
    \item Key hardware-software surface (cf. compilers)
    \item System management: bootstrap, shutdown, watchdogs
    \item Low-level abstractions and services
    \begin{itemize}
      \item Programming: processes, threads, IPC, program model
      \item Resource sharing: scheduling, multiplexing, virtualisation
      \item I/O: device drivers, local/distributed filesystems, network stack
      \item Security: authentication, encryption, permissions, labels, audit
      \item Local or remote access: console, window system, SSH
    \end{itemize}
    \item Libraries: math, protocols, RPC, cryptography, UI, multimedia
    \item Other stuff: system log, debugging, profiling, tracing
  \end{itemize}

\end{frame}

\begin{frame}
  \frametitle{Why study operating systems?}

  \bigskip

  The OS plays a central role in \textbf{whole-system design}  when building
  efficient, effective, and secure systems:

  \bigskip

  \begin{itemize}
    \item Key interface between hardware and software
    \item Strong influence on whole-system performance
    \item Critical foundation for computer security
    \item Exciting programming techniques, algorithms, problems
    \begin{itemize}
      \item Virtual memory; network stacks; filesystems; runtime linkers; ...
    \end{itemize}
    \item Co-evolves with platforms, applications, users
    \item Multiple active research communities
    \item Reusable techniques for building complex systems
    \item Boatloads of fun (best text adventure ever)
  \end{itemize}
\end{frame}

\begin{frame}
  \frametitle{FreeBSD}
  \begin{itemize}
  \item Open Source
  \item Unix
  \item Posix
  \item Complete System
  \item 20 years of history
  \end{itemize}
\end{frame}

\begin{frame}
  \frametitle{Overview}
  \begin{itemize}
  \item Today
    \begin{itemize}
    \item Introduction to DTrace
    \item Processes and the Process Model
    \item Scheduler
    \item Locking
    \end{itemize}
    \begin{itemize}
    \item Tomorrow
      \begin{itemize}
      \item Networking
      \item Filesystems
      \end{itemize}
    \end{itemize}
  \end{itemize}
\end{frame}

\end{document}

%%% Local Variables:
%%% mode: latex
%%% TeX-master: t
%%% End:
